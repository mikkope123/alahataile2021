\documentclass[../ala_hataile.tex]{subfiles}
\begin{document}
\raggedbottom
\clearpage
\includepdf[pages=22-23, pagecommand={}]{sisasivut_19062018.pdf}
\twocolumn[\section{Tilastotiede}]
\subsection*{Yleistä}
Tilastotiede tutkii laajassa mielessä aineistojen keräämistä, käsittelyä ja niistä tehtävää päättelyä. Keskeisessä roolissa tilasto\-tieteessä ovat aineistot, ja missä on aineistoja, on myös tilastotieteilijöitä, joten saatatkin jo pian huomata osaamisesi suuren tarpeen muun muassa taloudessa, lääke\-tieteessä, psykologiassa, biologiassa tai yhteis\-kunta\-tieteissä -- ihan vain pinta\-raapaisuna.

Opinnoissa lähdetään liikkeelle toden\-näköisyys\-laskennan ja tilastollisen päättelyn perusteista ja R-ohjelmoinnista. Aine\-opintoihin siirryttäessä kurssit syventyvät vahvemmin tieteen\-alan matemaattiseen teoria\-pohjaan. Toteutukseltaan monet opinto\-jaksot ovat perinteisiä luento\-kursseja, joita täydentävät viikoittaiset lasku\-harjoitukset, jotka voivat kurssista riippuen tapahtua joko paikan päällä ryhmässä tai kokonaan sähköisesti -- ohjausta on kuitenkin aina hyvin saatavilla niin verkossa kuin kampuksella. Vastapainona laskemiselle myös käytännön data-analyysillä on tilastotieteen osaajalle merkittävä rooli, joka palauttaa välillä korkealentoisen tuntuisista aiheista tiiviisti takaisin käytännön sovelluksiin. Juuri näiden sovellusalueiden takia oikeastaan vain taivas onkin rajana vapaasti valittavissa opinnoissa, sillä tilastotiedettä löydät lähes kaikkialta.

Tilastotieteilijöille on työelämässä paljon kysyntää ja työttömyysprosentti lähentelee nollaa. Työelämässä data on yhä kasvavassa roolissa, ja varsinkin jonkin verran tietojenkäsittelytiedettä tuntevalle tilastotiede tarjoaa datatieteen tehtäviin loistavan pohjan. Omien mielenkiinnonkohteiden mukaan onkin mahdollista sijoittua lähes tulkoon mille vain sektorille esimerkiksi erilaisiin tutkimus-, kehitys-, ja konsultointitehtäviin.

Muiden tieteenalojen opiskelijoille tilastotiede tarjoaa erittäin mielenkiintoisen kokonaisuuden, jossa jo ensimmäiset kurssit antavat paljon uutta ja hyödyllistä. Kynnys perusopintojen valitsemiseen on matala ja liikkeelle lähdetään aivan perusteista, joten omia esitietojaan ei tarvitse jännittää. Kokemuksen perusteella tähän aineeseen koukuttuu nopeasti ja halu oppia jatkuvasti enemmän vie helposti mukanaan myös aineopintojen puolelle. Tilastotieteilijöiden perinteisesti varsin kompaktista lukumäärästä johtuen merkittävä osa kurssien osallistujista on muiden tieteenalojen opiskelijoita, joten yksin jäämistä ei tässä aineessa tarvitse pelätä. 

\subsection*{Tilastotieteen kursseja}
\subsection*{Tilastotieteen perusopinnot}
\subsubsection*{Tilastotiede ja R tutuksi~I+II (5+5~op)}
Näillä kursseilla aloitetaan tilastojen,
tilastojen analyysin ja
todennäköisyyslaskennan opiskelu
perusasioista ja siirrytään 
syventäviin asioihin, kuten big dataan, Bayesin
kaavan sovellutuksiin muissa tieteissä sekä
regressioon ja luottamusväliin.

Ensimmäinen kurssi käsittelee sellaisia
asioita kuin keskeinen raja-arvolause, 
ehdollinen todennäköisyys ja
R-ohjelmointikielen opettelu. Nämä ovat tärkeimpiä
tilastotieteilijän työkaluja, eikä kurssilla
kannata tinkiä materiaalin lukemisesesta ja
koodaustehtävien tekemisestä, vaikkei
kurssikoetta olekaan. Kurssilla palautetaan
viikottain kuusi tehtävää, joihin kuuluu sekä laskuja
että RStudiolla tehtävää koodia. Nämä tehtävät
sinä ja kaksi kurssitoveriasi sitten arvostelette
esimerkkivastausten mukaan (tästäkin saa
pisteitä, joten arvostelu on hyvä tehdä) ja 
kurssin lopussa kuuden viikon
pisteet ynnätään yhteen ja summa päättää
kurssiarvosanasi. Jos tuntuu epäreilulta tai
vaikealta, niin kurssin ohjaajat auttavat ja
tarkistavat tilastollisesti merkittävästi
toisistaan poikkeavia arvosteluja ja kurssin lopuksi
voi myös valittaa arvosteluista, jos niillä on ollut
vaikutusta kurssiarvosanaan.

\subsubsection*{Todennäköisyyslaskenta~I (5~op)}
Todennäköisyyslaskenta~I on,
kuten nimestä saattaa arvata, lyhyt johdatus
todennäköisyyslaskennan perusteisiin.
Kurssilla käydään läpi huomattavasti erilaisia
todennäköisyyslaskennan saralla tärkeitä
työkaluja, mutta todistukset jätetään
kurssilla vähemmälle.

Kurssista tuleekin mieleen jossain määrin
lukiokurssi, sillä pää\-määränä on oppia
käyttämään toden\-näköisyys\-laskentaa, ei
niinkään ymmärtää sen matemaattisia perusteita.
(Tämä johtuu aika voimakkaasti
siitä, että toden\-näköisyys\-laskennan matemaattinen
pohja nojaa mitta\-teoriaan, johon
tutustutaan vasta kurssilla Mitta ja integraali.)
Kussilla ei ole erityisesti esi\-tieto\-vaatimuksia
ja sen voi käydä hyvin ensimmäisen
vuoden keräällä. Kurssi kestää vain yhden
periodin, mutta seuraavassa periodissa
luennoitava johdatus tilastolliseen päättelyyn
on hyvä jatke kurssille.

\vspace{0.5cm}
\noindent\textsc{Rami Luisto}

\subsubsection*{Tilastollinen päättely~I (5~op)}
Kurssi ei
varsinaisesti paneudu tilastollisen päättelyn
matemaattiseen teoriaan. Sen sijaan siinä
tutustutaan tilastotieteen peruskäsitteistöön
ja "-periaatteisiin. Kurssin jälkeen opiskelija
ymmärtää tilastollisen päättelyn formaalin
perustan (tai sen olemassaolon) sekä tilastollisten
tulosten oikeat tulkintatavat.
Kurssi sopii erinomaisesti opiskeltavaksi
heti Johdatus todennäköisyyslaskentaan
"-kurssin jälkeen. Monille opiskelijoille
tämä jää ainoaksi tilastotieteen kurssiksi.
Heille kurssi antaa tietoja, joita tulee varmasti
tarvitsemaan niin arkielämässä kuin
tutkimustyössä. Tilastotieteestä enemmän
kiinnostuneille kurssi tarjoaa ``pehmeän
laskun'' teoreettisempiin tilastotieteen
kursseihin kuten Tilastolliseen päättelyyn.

\vspace{0.5cm}
\noindent\textsc{Yilong Li}

\subsubsection*{Data-analyysin projekti (5~op)}
Tällä kurssilla tehdään itsenäinen tutkimus, jossa analysoidaan avoimen datan raaka-aineistoa. Kurssi kannattaa käydä perusopinnoista viimeisenä, sillä siellä on R-ohjelmiston lisäksi osattava myös muita tilastotieteen perusteita.

\subsection*{Tilastotieteen aineopintokursseja}
\subsubsection*{Todennäköisyyslaskenta~IIa/b (5+5~op)}
Todennäköisyyslaskenta IIa ja b muodostavat Matematiikan perusopintojen ohella tarvittavan teoreettisen pohjan tilastotieteen opinnoille. Tämän vuoksi kahden kurssin muodostama kokonaisuus onkin selkeästi vähemmän soveltava ja käytännönläheinen kuin Todennäköisyyslaskenta~I. Toden\-näköisyys\-laskenta~IIa ja b ovat pakollisia tilasto\-tieteen ja ekonometrian opinto\-suunnissa, ja ne suositellaan suoritettaviksi toisen opinto\-vuoden syksynä.

\vspace{0.5cm}
\noindent\textsc{Daniel Kari}
\subsubsection*{Tilastollinen päättely~II (10~op)}
Kurssilla syvennytään perusopinnoissa esiteltyihin tilastollisen päättelyn keskeisimpiin käsitteisiin ja esitellään monia työkaluja, joita jokainen tilastotieteen soveltaja tarvitsee. Kurssi on sisällöltään huomattavasti teoreettisempi kuin Tilastollinen päättely~I. Kurssi on pakollinen tilastotieteen ja ekonometrian opintosuunnissa ja suositellaan suoritettavaksi toisen opintovuoden keväänä kurssin Todennäköisyyslaskenta~II jälkeen.

\vspace{0.5cm}
\noindent\textsc{Daniel Kari}
\subsubsection*{Lineaariset mallit~I--II (5+5~op)}
Lineaariset mallit ovat olennainen osa tilastotieteilijän työkalupakkia. Kurssilla tutustutaan lineaaristen mallien perusteoriaan ja esimerkiksi niiden rakentamiseen ja hypoteesien testaamiseen. Kurssi on pakollinen tilastotieteen ja ekonometrian opintosuunnissa ja suositellaan suoritettavaksi toisen opintovuoden keväällä Todennäköisyyslaskenta~II ja Tilastollinen päättely~II jälkeen. Suositeltua on myös suorittaa pohjalle Lineaarialgebra ja matriisilaskenta~III. Toteutukseltaan kurssi on perinteinen luennot ja viikoittaiset laskuharjoitusryhmät "-kombo.

\vspace{0.5cm}
\noindent\textsc{Anna Holvio}
\subsubsection*{Bayes-päättely (5~op)}
Kurssi käsittelee tilastollista päättelyä bayesiläisittäin sekä analyyttisesti että simuloimalla. Alkuun bayesiläinen päättely saattaa aiheuttaa hämmennyksen tunteita, mutta kurssin edetessä kaavat käyvät jo niin tutuiksi, että painuvat kyllä mieleen. Mielenkiintoisen lisän kurssiin tuovat perinteisten käsin tehtävien laskuharjoitusten lisäksi R- ja Stan-ohjelmistoilla tehtävät simulointi\-harjoitukset. Kurssi on pakollinen tilastotieteen opintosuunnassa ja suositellaan suoritettavaksi kolmannen vuoden syksyllä. Kurssin opetuskielenä on englanti.

\vspace{0.5cm}
\noindent\textsc{Anna Holvio}
\end{document}