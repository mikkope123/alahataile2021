\documentclass[../ala_hataile.tex]{subfiles}
\begin{document}
\raggedright
\textbf{AALTOYLIOPISTO}: Suuri innovaatio,
jossa taiteilijat voivat keksiä hienoja juttuja,
tekniikan tietäjät rakentaa niitä ja bisnesihmiset
sitten myydä.

\textbf{AATU}: Akateeminen Aurajokilaivuritutkinto.
Kevätlukukauden päättymistä juhlistava
opiskelijatapahtuma toukokuussa
Suomen Turussa.

\textbf{AEE}: Mukava naapuri Klusterilla. Tykkää
erityisesti opiskelijoista.

\textbf{AIESEC}: mm.\,käpistelijöiden ja taloustieteilijöiden
kansainvälinen opiskelijavaihtojärjestö.

\textbf{AIKATALO}: Mikonkatu~8, Ateneumin
takana.

\textbf{AINEJÄRJESTÖHÖRHÖ}: Ihminen,
jolle ainejärjestö(t) ovat ainakin osaksi
elämäntapa. Useamman vuoden ainejärjestöaktiivi.
Saattaa johtaa ainejärjestöaddiktioon.

\textbf{AIRA}: (Fys.sl.) Aineen rakenne (I ja II).

\textbf{AKATEEMINEN VARTTI}: Luennot ja
laskarit alkavat 15 minuuttia yli tasatunnin,
senkun kaikki kuitenkin myöhästyisivät.
Opiskelijoiden hämäämiseksi tentit alkavat
kuitenkin aina tasalta.

\textbf{AKATEEMINEN WARTTI}: Juostiin
vapun jälkeen, nykyisin ties milloin.
OLL:in järjestämä leikkimielinen liikuntatapahtuma.

\textbf{AKATEEMINEN VAPAUS}: Illuusio,
joka on joskus kuulemma ollut tottakin.
\textbf{ALASAUNA}: Saunatila Uuden Ylioppilastalon
B-rapun kellarissa.

\textbf{ALEKSANDRIA}: Kaupunki Egyptissä.
Myös opiskelijakirjastosta, Kielikeskuksesta
ja ATKtiloista muodostuva oppimiskeskus
Vuorikadun ja Fabianinkadun välissä.

\textbf{ALINA-SALI}: Juhlasali Uuden yo-talon
3.\,kerroksessa. Tunnettu bileiden pitopaikka.

\textbf{ALMA}: Yliopiston vanhempi intranet. Sisältää
nykyisin joitakin harvoja toimintoja
joita ei flammasta löydy.

\textbf{ALTER EGO}: Yliopiston roolipelikerho.
Julkaisee Alterations-lehteä, pitää scifi- ja
fantasiavideoiltoja, järjestää roolipelitoimintaa.

\textbf{AMANUENSSI}: Hallintovirka, virallinen
nimitys orjatyövoimalle. Tekee hommat
sillä aikaa, kun itse pääjehu polttaa kessua.

\textbf{AMOR}: Roomalaisten rakkauden jumala.
(Kem.sl.) Atomien ja molekyylien rakenne.

\textbf{APPRO}: 1) Approbatur. Vanha perusopintokokonaisuus.
2) Erityisesti Idan Appro
ja/tai Kumpulan Appro. 3) Myös merkitsemässä
baarikierroksia, kuten Helsinginkadun
(Hesarin) appro, Hämeenkadun appro,
Limeksen appro

\textbf{ASSARI}: Assistentti tai tuntiopettaja. Pitää
laskareita, luentoja, labroja ja päivystää.
Saa koskea, kysyä, sinutella ja muutenkin
vaivata. Neuvoo mieluiten vastaanottoaikana.

\textbf{ATK-ASEMA}: Yliopiston ATK-osaston ylläpitämiä paikkoja, joissa voi käyttää
tietotekniikkaa veloituksetta (tarvitset ADtunnuksen).
Mikroja, sovellusohjelmia,
lasertulostimia, skannereita, nopeat verkkoyhteydet\dots

\textbf{BOTTA}: Pohjalaisten osakuntien omistama
bailaushelvetti Museokadulla.

\textbf{CASA}: Casa Academica. Keskisuomalainen
osakunta ja Hankenin biletila. Dommaa
vastapäätä.

\textbf{CITY-KÄYTÄVÄ}: Suorin reitti rautatieasemalta
Stokkalle.

\textbf{CIVIS}: (Lat.). Osakuntatermi: vanha opiskelija.
Ks.\,fuksi.

\textbf{COCA-COLA, COLA}: Käpistelijöiden
(ja miksei muidenkin) keskuudessa kovin
suosittu janojuoma, auttaa kuulemma koodaamista.

\textbf{CUMU}: Cum laude approbatur, aineopinnot.

\textbf{DEMO, DEMONSTRAATIO}: (Fys. ja
kem.sl.) Esiintyy mm.\,peruskursseilla, jolloin
kaksi assaria yrittää leikkiä erilaisilla
hauskoilla vempeleillä vaihtelevalla menestyksellä.

\textbf{DEMOEFEKTI, DEMOILMIÖ}: Jos
jokin voi mennä pieleen, se menee. Erityisesti
esiteltäessä ohjelman tai tietokoneen
toimintaa.

\textbf{DIFFIS}: (Mat.sl) Matematiikan kurssit
Differentiaaliyhtälöt I ja II.

\textbf{DOMMA}: Domus Academica (Opiskelijoiden
Koti): osoite Hietaniemenkatu~14 tai
Leppäsuonkatu~7--9. Asuntojen lisäksi ennen
myös järjestötiloja; Eritoten vanhojen
puheissa DC ja $-2$ (``miinus-2''), vanhoja
biletiloja.

\textbf{DOMUS GAUDIUM}, DG: Ilontalo, kolmas
ylioppilastalo, jonka tunnuslause on
kierosti ``sub hoc tecto cives academici excoluntur'',
so.\,``tämän rakennuksen suojissa
tehdään akateemisia kansalaisia''. Limeksen
ja lukuisien muiden järjestöjen kerhohuoneet
ovat täällä, muista ovikello!

\textbf{DÖSÄ}: Bussi, linja-auto, onnikka, HKL:n
sininen.

\textbf{EDARI, EDUSTAJISTO}: HYYn ylin
päättävä elin (vrt.\,eduskunta): vaalit joka
toinen vuosi. vaihtuvissa paikoissa, myös
kampuksilla. Vapaa pääsy.

\textbf{EETTERIPYÖRTEET}: Mullistava teoria,
joka selittää kaiken alkuräjähdyksestä
ja luomisesta lähtien. Ks.\,Nieminen.

\textbf{EGEA (European Geography Association)}:
Kansainvälinen maantieteilijöiden
järjestö, jonka Helsingin jaostossa toimii
aktiiveina useita ahkeria mantsalaisia.

\textbf{ELIELINAUKIO}: Postitalon ja Rautatieaseman
välissä oleva aukio, jolta lähtee iso
liuta busseja. Saunailtojen yhteislähtö on
usein täältä.

\textbf{EPSILON}: Matemaatikkojen jumala.

\textbf{ESITISLE}: HYKin painotuote.

\textbf{ESN}: Erasmus Student Network. Vaihtoopiskelijoiden
vastaanottoa ja orientoimista
hoitava kansainvälinen järjestö. Suomen
osasto on HYYn ESN-valiokunta.

\textbf{ESPA}: Esplanadi, pohjois- ja etelä-, välissä
puisto, josta löytää patsaita, nuorisoa ja
turisteja sekä tietysti Kappelin ja ravintola
Teatterin.

\textbf{EURO}: Vaihdon väline (eli fyffe, raha
yms.), joka on käytössä n.\,17 Euroopan
maassa.

\textbf{EXCU(RSIO)}: Ekskursio; retki, joka kestää
muutamasta tunnista $n$:ään vuorokauteen.
Sivistää tieteellisesti, taiteellisesti tai
muuten vain.

\textbf{EX-TEMPORE -LASKARIT}: (Mat.
ja fys.sl.) Laskuharjoitus, jonka tehtäviä
ei anneta etukäteen, vaan ne ratkaistaan
paikan päällä yksin tai pienissä ryhmissä.
Joillakin kursseilla kutsutaan myös ohjauksiksi.

\textbf{FILOSOFINEN TIEDEKUNTA}: Muinaismuisto,
joka toimii lähinnä promootiojärjestelyjä
varten. Elää tosin vielä tutkintonimikkeissä.
Aikanaan ML-tiedekuntakin
kuului osana.

\textbf{FIL. YO}: Filosofian ylioppilas. Meille
opiskelemaan päässeen snobbailua veroilmoituksissa,
työhakemuksissa, puhelinluettelossa,
ym\dots

\textbf{FL}: Filosofian lisensiaatti. FM:n ja FT:n
väliinputoaja.

\textbf{FLAMMA}: Yliopiston oma intranet, erilaisia
kattavia tietopaketteja aina työnhausta
opiskeluun.

\textbf{FM}: Filosofian maisteri, ylemmän akateemisen
loppututkinnon suorittanut henkilö.

\textbf{FORUM}: Vrt.\,Forum Romanum. Kauppakeskus
ylioppilastaloa vastapäätä.

\textbf{FT}: Filosofian tohtori. Harvat ja valitut.

\textbf{FUKSI}: (saks.\,fuchs). Aloittelija, uusi
opiskelija, keltanokka, phuksi.

\textbf{FUKSISUUNNISTUS}: Ei fuksien mopotusta,
vaan rento kaupunkisuunnistus
fukseille, fuksiryhmille ja fuksinmielisille.
Rastit eri puolilla keskustaa, tehtävät monensorttisia
ja fuksin älyä ja mielikuvitusta
mittaavia. Päättyy yleensä Fuksiaisbileisiin.

\textbf{FYMM}: (Fys.sl) Fysiikan matemaattiset
menetelmät (Ia, Ib, IIa, IIb \& III) Teor. fys.
kurssit.

\textbf{GEYSIR}: Geofyysikkojen ainejärjestö.

\textbf{GIS (Geographic Information System)}:
Geoinformatiikka, maantieteen suuntautumisvaihtoehto.

\textbf{GRADU}: Syventävien opintojen tutkielma, eli maisteriksi valmistuvien lopputyö.
Pro Gradu -- tuttavallisesti vain ``Iso G''
tai ``G''. Ei välttämättä ole kohteliasta mennä
kysymään ainejärjestöaktiivilta kuinka
hänen gradunsa jakselee.

\textbf{GURULA}: TKO-älyn opiskelijahuone
TKT:llä, vrt.\,Gurun luola

\textbf{HAALARIT}: Opiskelijoiden konttausasu
vappuisin ja muulloinkin. Matemaatikoilla
Matrixin kirkkaanpunaiset, fyysikoilla Resonanssin
fuksianpunaiset, käpistelijöillä
TKO-älyn kirkkaankeltaiset, meteorologeilla
Synopin auringonkeltaiset, kemisteillä
HYKin kirkkaanmustat, tähtitieteilijöillä Meridiaanin yönsiniset sekä
geofyysikoilla Geysirin vihreänharmaat.

\textbf{HALLINTORAKENNUS}: Porthaniaa ja
päärakennuksen uutta puolta vastapäätä: sisältää
mm.\,urheilutiloja kellarissa.

\textbf{HALLITUS}: Toimeenpaneva ja valvova
hallintoelin. Vrt.\,Suomen hallitus, erit.\,Limeksen
hallitus. Myös HYYn.

\textbf{HALLOPED}: Hallinnon opiskelijaedustaja, edunvalvojasi yliopistolla. Opiskelijat ovat edustettuina mm.\,kandi- ja maisteriohjelmien johtoryhmissä, tiedekuntaneuvostossa, yliopistokollegiossa sekä yliopiston hallituksessa. Mikäli sinulla on kehitysehdotuksia opetussuunnitelmiin tai et ymmärrä miksi asiat on tehty niin vaikeiksi yliopistolla, ensimmäinen yhteyshenkilö on oma hallopedisi.

\textbf{HANKEN}: Svenska Handelshögskolan.
Lähellä Dommaa, kukaan meistä ei vielä
ole uskaltautunut sisään, ruoka on kuulemma
hyvää. Huhujen mukaan opiskelemaan
pääsee jos osaa laskea ruotsiksi kymmeneen.

\textbf{HANSAKUJA, HANSATORI}: Citykorttelin
HYYn omistaman pään sisäistä
maantietoa.

\textbf{HAO}: Helsingin Aineenopettajiksi Opiskelevat,
jokaisen opelinjalaisen oma ainejärjestö.

\textbf{HEILAHTELEEPYÖRÄHTELEE}:
Ks.\,Nieminen ja eetteripyörteet.

\textbf{HELIX}: Helix = biokemian sekä solu- ja
molekyylibiologian opiskelijoiden ainejärjestö

\textbf{HESARI}: 1) Helsingin Sanomat 2) Helsinginkatu
(Kalliossa, paljon nestetankkauspaikkoja);
Erityisesti Hesarin appro

\textbf{HIEKKALAATIKKO}: Lasten ja lastenmielisten
leikkipaikka. Myös parveke
Unicafen yläpuolella Physicumin kolmoskerroksessa.

\textbf{HIETSU}: Hietaniemen uimaranta Hietaniemen
hautausmaan kupeessa. Joskus
myös Hietsun kirppis Hietalahdentorilla.

\textbf{HIIT}: Helsinki Institute for Information
Technology, informaatioteknologian tutkimuslaitos.

\textbf{HIP}: Helsinki Institute of Physics, fysiikan
tutkimuslaitos.

\textbf{HOAS}: Helsingin seudun opiskelija-asuntosäätiö.
Kilteille opiskelijoille halpoja
asuntoja kalliiseen hintaan.

\textbf{HT}: Harjoitustehtävä / hyvin triviaali /
helppo todistaa, jotain muutaman rivin laskusta
puolen päivän pähkäilyyn. Prujuista
/ luentokalvoista pois syystä tai toisesta jätetty
osa. Usein esim.\,hyvä aihe gradulle.

\textbf{HUMANISTI, HUMANOIDI}: Humanistisessa
tiedekunnassa opiskeleva.

\textbf{HY}: Helsingin yliopisto (HU på svenska).

\textbf{HYAL}: Helsingin Yliopiston Ainejärjestöläiset.
Eri tiedekuntien ainejärjestöjen yhteenliittymä.
Ks.\,Edari/Edustajisto.

\textbf{HYK}: Helsingin yliopiston kemistit; usein
tekemisissä Limeksen kanssa.

\textbf{HYLSY}: Hylätty. Ei mennyt tentti läpi tällä
kertaa.

\textbf{HYPPY}: Koordinoitu lihasten liikesarja,
jolla pyritään voittamaan maan painovoima
ja irtautumaan maasta vaihtelevalle
korkeudelle. Myös Helsingin yliopiston
polymeeri- ja puukemistit.

\textbf{HYRMY}: Helsingin yliopiston raskaan musiikin ystävät. Raskaaamman musiikin
ystävien oma yhdistys

\textbf{HYROKRAATTI, HYYPIÖ}: Yliopistolta
tai HYYstä suojatyöpaikan saanut
ongelmalapsi tai poliitikon uralle aikova
yleensä enemmän tai vähemmän kirkassilmäinen
opiskelija.

\textbf{HYSFK}: Yliopiston sci-fi klubi. Järjestää
mm.\,videonäytöksiä ja julkaisee Marvin --
The lehteä.

\textbf{HYTKY}: Helsingin yliopiston teknokulttuurin
ystävät. Järjestää koneelliseen nykymusiikkiin
perehdyttäviä bileitä.

\textbf{HYY}: Helsingin yliopiston ylioppilaskunta:
Mansku 5 A--C, suurin ja mahtavin.
Valmistaa hyrokratian maistereita, omistaa
yhtä sun toista, mm.\,Kaivotalon ja Kilroyn.

\textbf{HÄMIS}: Hämäläis-Osakunta. Majailee
punatiilirakennuksessa Kampin keskuksen
vierellä Urho Kekkosen kadulla.

\textbf{HÖRHÖ}: Henkilö, joka hörhöilee. Vrt.\,ainejärjestöhörhö.

\textbf{HÖRHÖILLÄ}: Harhailla vailla selkeää
päämäärää usein tehden ainakin muiden
mielestä naurettavia asioita. Muiden kuin
hörhöjen mielestä hörhöily on yleensä turhaa
tai jopa ärsyttävää.

\textbf{IAPS}: International Association of Physics
Students. Fysiikan opiskelijoiden kansainvälinen
yhteistyöjärjestö.

\textbf{ICPS}: International Conference for Physics
Students, jokavuotinen viikon kestävä
fyysikkonörttien kokoontuminen.

\textbf{IIDA}: Opiskelijakämppäläkiinteistö Pohjois-Haagassa, Ida Aalbergin tie~1:ssä.
Sisälsi vuoteen~2015 saakka opiskelijajärjestöjen
paljon käyttämän saunatilan. Ks.\,saunailta.

\textbf{IIDAN APPRO}: Iidan saunailloissa suoritettava
korttelin ympärijuoksu Aatamin
asussa ja vähän Eevankin.

\textbf{IIDAN CUMU}: Viuhahduksen laajempi
oppimäärä. Yleensä Pohjois-Haagan ostarille
ja takaisin.

\textbf{IIDAN LAVI}: Raskaan sarjan viuhahtajille:
Pohjois-Haagan asemalle ja takaisin.
Muista sivuaineet ja mahdolliset muut jatko-opinnot.

\textbf{ILOTALO}: Iloinen paikka viettää aikaa,
so.\,syntiä. Ks.\,Domus Gaudium.

\textbf{IRC}: Syntiä.

\textbf{ISIC}: (1)~Muinais\-egyptiläinen jumalatar. (2)~Islamilainen valtio, jonka lippu\-design on kopioitu Limekseltä. (3)~Kansainvälinen opiskelijakortti.

\textbf{JOHTORYHMÄ}: Kandi- ja maisteriohjelmien päättävä
elin, jossa on proffien ja henkilökunnan lisäksi
myös opiskelijaedustus. Hyvät mahdollisuudet
vaikuttaa opetussuunnitelmiin, opetusohjelmaan
ym.\,opetusta koskevaan.

\textbf{KAAPELITEHDAS}: Monitoimitalo
Ruoholahdessa, entinen Nokian kaapelitehdas.
Sisältää taidetta, teatteria, näyttelyitä,
tapahtumia, bileitä, elokuvia ym.

\textbf{KAISANIEMI}: Puisto Rautatieaseman ja
Pitkänsillan välissä. Erityisesti Kaisiksen
jalka- ja pesäpallokenttä.

\textbf{KAIVARI}: Kaivopuisto. Ylioppilaskuntienyhteinen
pyllymäki laskiaisena. Ks.\,Ullanlinnanmäki.

\textbf{KAIVOKÄYTÄVÄ}: Kaivotalon sisäistä
maantietoa.

\textbf{KAIVOPIHA}: Aukko Kaivotalossa.

\textbf{KAIVOTALO}: HYYn omistama kiinteistö,
joka piirittää Uutta ylioppilastaloa
suunnalta jos toiselta.

\textbf{KAPPELI}: Kahvila Espan puistossa.

\textbf{KAPTEENI LIMES}: LSP:n luoma supersankari,
yli-inhimillinen limetti, jonka
tehtävänä on suojella järjestöä ulkoisilta
vaaroilta (mm.\,Tupsulakeilta, Humanoideilta
ja Krapulamieheltä).

\textbf{KASVIS}: (1) Kasvihuone, eli geologian opiskelijoiden
taukotila Physicumissa. (2) Kasvatustieteellinen tiedekunta.

\textbf{KEKKONEN}: Edesmennyt Suomen presidentti.
Myös edesmennyt saunatila Kaivokadulla.

\textbf{KERTSI}: kts.\,Klusteri

\textbf{KESKUSPUISTO}: Viheralue Töölönlahdelta
Jäämerelle: kelpaa pyöräilyyn, kävelyyn,
hölkkäilyyn, istuskeluun tai vaikkapa
ratsastukseen.

\textbf{KEVÄTREKI}: Limeksen jokakeväinen
invaasio/picnic Suomenlinnaan. Luullaan
yleensä painovihreeksi. EI silti ole Kevätretki,
niin kuin kylläkin mainoksessa aikanaan
piti olla.

\textbf{KIASMA}: Nykytaiteen museo Mannerheimintiellä
postin vieressä.

\textbf{KIELIKESKUS}: Fabianinkatu~26: sisältää
itse\-opiskelu\-studion ym.\,kieliin liittyvää.

\textbf{KILROY TRAVELS}: Matkatoimisto
Kaivopihalla ja muuallakin maailmassa.
Matkoja opiskelijoille ja muillekin.

\textbf{KLUSTERI}: Kerhohuone, Limeksen ja
muiden Matlun järjestöjen oma sijaitsee
osoitteessa Mechelininkatu~3~C. Vapaa pääsy,
jos paikalla vain on joku avaimenhaltija.
Ohjelmaa on vilkkaimmillaan päivittäin:
Limeksen lukuisat kerhot ja puuharyhmät
kokoontuvat täällä. Myös epävirallista ohjelmaa
ja hengailua.

\textbf{KLUUVI}: Forumin vastine yliopistokorttelissa.
McDonald's ym.

\textbf{KOLMEN SEPÄN PATSAS}: Vanhan ja
Stockan välisellä aukiolla (Kolmen sepän
aukio) oleva siveettömyydellään suurta kohua
herättänyt patsas.

\textbf{KOMERO}: Matemaagikkojen huone
Exactumissa.

\textbf{KOULUTUS}: Älä alistu koulutettavaksi.
Vain koiria ja kadetteja koulutetaan.

\textbf{KULTSA}: (Limes-sl.) Limeksen kulttuurijaosto.
Limeksen kulttuurisihteerikin on
Kultsa. Järjestää excursioita ja tapahtumia
limesläisille.

\textbf{KUMA}: Kulttuurimaantiede.

\textbf{KUMARETKI}: Keväällä järjestettävä
kulttuurimaantieteellinen opintoretki maolaisille, suuntautuu yleensä Pohjoismaihin,
Baltiaan tai Venäjälle.

\textbf{KUMPULA}: ML-tiedekunnan osastot (lukuun\-ottamatta Kaupunki\-tutkimus\-instituuttia)
sijaitsevat täällä. Reilut 20~vuotta suunniteltiin
yhtenäistä kampusaluettta ja vihdoin
vuonna~2004 toteutui kokonaisuudessaan.
Kemistit muuttivat ensimmäisinä
Kumpulaan vuoden~1995 alusta, yleinen
toimisto ja opintotoimisto elokuussa~1996.
Perässä tulivat fyysikot vuonna~2001 ja matemaatikot
sekä käpistelijät vuonna~2004. Käpistelijät muuttivat Kumpulaan junalla; vanha ratapohja kulkee edelleen Vallilan\-laakson reunalla.

\textbf{KURKI-SUONIOT}: (Fys.\,sl.) Legendaariset
fysiikan peruskurssien oppikirjat, joita
edelleen käytetään selventämään englanninkielisiä
tiiliskiviä.

\textbf{KV}: Kansainvälinen ("-työryhmä, "-jaosto
ym.) mm.\,Limeksen KV-toiminta.

\textbf{KY}: Helsingin kauppa\-korkea\-koulun ylioppilaat~ry, kyltereiden oma yhdistys.

\textbf{KYLTERI}: Kauppatieteiden opiskelija.

\textbf{KYYKKÄ}: Suomalainen perinne\-urheilu\-laji, jossa suistetaan puupalikoita (kyykkiä)
isommalla puupalikalla (kartulla) ulos vastustajan pelialueelta. Akateemiset MM-kisat
helmikuussa.

\textbf{KÄPISTELIJÄ}: (TKT.\,sl.) TKT:n opiskelija.

\textbf{KÄPISTELY}: (TKT.\,sl.) TKT:n opiskelu.
Ks.\,käpistelijä.

\textbf{KÄYTÄVÄVOHVELIT}: Joka syksy ja
kevät järjestettävä tapahtuma, jossa maantieteilijät
paistavat vohveleita kaikkien
Kumpulalaisten iloksi.

\textbf{LABRA}: Laboratorio, "-työ. Piikki opiskelijan
lihassa. Periaatteessa hyödyllinen
keksintö, jossa on jopa omatoimisen ajattelun
ja oppimisen vaara.

\textbf{LOIMU}: Luonnon-, ympäristö- ja metsätieteilijöiden liitto Loimu~ry.

\textbf{LAMBDA}: Kreikkalaisten aakkosten 11.\,kirjain, kuulun tietokonepelin tunnus ja fysikaalinen aallonpituus. Myös teoreettisen
tietojen\-käsittelytieteen opiskelijoiden ryhmä.

\textbf{LAMMI}: Nyk.\,Hämeenlinnan osa, jonka
biologisella asemalla järjestetään jokakeväinen
maantieteen kenttäkurssi.

\textbf{LASIPALATSI}: Kulttuuri- ja kahvilakeskittymä
Kampin ja Mannerheimintien välissä.

\textbf{LASKARIT}: (Lasku)harjoitukset. Yleensä
vähemmän tai enemmän pakollisia, mutta
sitäkin hyödyllisempiä, sanovat.

\textbf{LAVI}: Laudatur, syventävät opinnot. Kaukainen
tavoite.

\textbf{LEMMA}: Apulause. Assari kiskoo hihasta
kun ei muuta keksi.

\textbf{LEPPÄTALO, LEPPIS}: Kolmannen
ylioppilastalon työnimi, nykyään Domus
Gaudium.

\textbf{LI**KE:, LixxxxKe}: Standardiformaatti
Limeksen lukemattomille eri kerhoille (LImeksen se ja se KErho). Esim. LiEKe tai
LiStraKe, sekä tusinoittain muita enemmän
tai vähemmän aktiivisia. Yleisperiaate:
kun mahdollisen kerhon nimi kerran on
lausuttu ja joku sen toistaa, sellainen on
siitä pitäen olemassa.

\textbf{LIEKE}: LImeksen ElokuvaKErho. Ilmaisia
elokuvia jäsenistölle säännöllisissä videoilloissa
ja "-öissä.

\textbf{LIJAKE}: Limeksen Jaloviina-kerho. Kokoontuu
usein kys.\,jalojuoman tiimoilta

\textbf{LIHAKE}: Limeksen HAsselhoff-KErho.
Hedonistinen Hasse-sedän palvojayhteisö,
järkkää mm.\,excuja seksimessuille.

\textbf{LIMAKE}: Limeksen matkailukerho. Matkoja,
useimmiten ulkomaille, mielellään
innostaviin ja/tai eksoottisiin kohteisiin.
Huomaa eroava kirjoitusasu Limake!

\textbf{LIMES}: Rajavalli, "-muuri. Erit.\,matematiikassa
raja-arvo. Maailmankaikkeuden
suurin ja kaunein ainejärjestö Helsingin
yliopistossa.

\textbf{LIMETTI}: Lime-hedelmä, vihreä sitruspallero.
Myös limesläinen, Limeksen jäsen.

\textbf{LINIS}: (Mat.\,sl.) Lineaarialgebran kurssien
yleisnimi.

\textbf{LINUX}: Meidän tietojenkäsittelytieteilijän
Linus Torvaldsin kehittämä ilmainen
UNIX-käyttöjärjestelmä. Valinnainen käyttis
yliopiston koneilla.

\textbf{LIPASTO}: Yliopisto.

\textbf{LIPPUPALVELU}: Myy ennakkolippuja
mm.\,Stokkalla ja Sokoksella. Myös puhelimitse
ja www-tilauksena.

\textbf{LISURI}: Lisensiaattityö. Odottaa lisuriksi
aikovia.

\textbf{LSP}: Limeksen Salainen Poliisi. Kukaan ei
tiedä, ketkä siihen kuuluvat -- eivät edes ne,
jotka kuuluvat.

\textbf{LUK, LUKKI, LUTKA}: Luonnontieteen kandidaatti,
alemman korkeakoulututkinnon
suorittanut luonnontieteilijä.

\textbf{LUMA}: Luonnonmaantiede.

\textbf{LYYRA}: Ylioppilaskunnan hieno uudis\-rakennus Hakaniemessä. Tieteen ja talouden kortteli ja pöhinäkeskus. Valmistunee ennen sinua.

\textbf{MAKKARATALO}: Rautatieasemaa vastapäätä
oleva parkki- ja liiketalo. (Katso,
niin näet ne makkarat).

\textbf{MANNERHEIM-SALI}: Uuden Ylioppilastalon
5.\,kerroksessa. Lähinnä kokouskäytössä.

\textbf{MANSKU}: Mannerheimintie, katu Helsingin
keskustasta jonnekin susirajan suuntaan.

\textbf{MANTA}: Alaston nainen, jolle annetaan
lakki lämmikkeeksi kun ilmat alkavat lämmetä. Espan toisessa päässä.

\textbf{MANTU}: MaOn julkaisema lehti, ilmestyy
neljä kertaa vuodessa.

\textbf{MAO}: Kiinan kommunistisen puolueen
pitkäaikainen johtaja. Myös maantieteen
opiskelijoiden ainejärjestö.

\textbf{MAPU}: (Fys.\,sl.) Matemaattiset apuneuvot.

\textbf{MARMORIKUJA}: Madonkolo Makkaratalossa.

\textbf{MATEMAAGIKKO}: (Mat.\,sl.) Matematiikan
opiskelija.

\textbf{MATKAKORTTI}: Kortti, jolle ladataan
aikaa (kautta) tai rahaa (arvoa). Tällä maksetaan
liput HSL:n liikenteessä. Välttämätön
väline julkisilla matkustettaessa.

\textbf{MATLU}: Matemaattis-luonnontieteellisten ja bio- ja ympäristötieteellisten
ainejärjestöjen yhteistyöjärjestö, perustettu
keväällä 1994.

\textbf{MATRIX}: (Lat.) Matriisi, luvuista muodostettu
taulukko, jossa $m$ riviä ja $n$ saraketta.
Elokuva, ja myös matematiikan ja
tilastotieteen opiskelijoiden ainejärjestö.

\textbf{MEGAZONE}: Kaikenikäisten lasten lasertaistelupeli. Parikymmentä ihmistä räiskimässä
toisiaan laseraseilla, joiden osumat
rekisteröidään. Suosittua ajanvietettä opiskelijajärjestöissä.

\textbf{MERIDIAANI}: Tähtitieteen opiskelijoiden
ainejärjestö.

\textbf{META}: Muut Esille Tulevat Asiat. Vapaamuotoista
nahistelua kokouksissa.

\textbf{METSÄTALO}: Fabianinkatu~39 / Unioninkatu~40. Kielitieteilijöitä. UniCafe kellarissa.

\textbf{ML}: Matemaattis-luonnontieteellinen tiedekunta.

\textbf{MOFY}: (Fys. sl) Teor.\,fys.\,kurssi (Moderni
fysiikka).

\textbf{MOKA}: Töppäys, virhearviointi, "-suoritus.
(Fys.sl) Teor.\,fys.\,kurssi (monen kappaleen
ilmiöt).

\textbf{MOODI}: Tilastotieteen opiskelijoiden ainejärjestö.

$n$: Erityisesti matemaattinen termi
eräälle suuruudenhullulle muuttujalle,
joka alati pyrkii kohti ääretöntä. Ks.\,$n$:nnen
vuoden opiskelija.

\textbf{$N$:nnen VUODEN OPISKELIJA}: Kauemmin
kuin viisi vuotta yliopistolla viihtynyt.

\textbf{N.N.}: Kuuluisa teor.\,fys.\,luennoitsija, monialainen
ihme.

\textbf{NAK}: Helsingin nuorisoasiainkeskus. Lainaa
järjestöille av-materiaalia (kuten videotykkejä)
ja paljon muuta.

\textbf{NAKKI}: (järjestösl.) Jokin suoritettava
tehtävä. Myös nakittaa: vierittää vastuu
hommasta jollekin toiselle.

\textbf{NIEMINEN}: Kauko Armas, varanotaari.
Eetteripyörreteorian isä ja uranuurtaja.
Lähde: Kauko Nieminen, Eetteripyörteet
voimina, 1984, ISBN-951-99532-4-8.

\textbf{NOPPA}: ks.\,opari

\textbf{OLL}: Opiskelijoiden liikuntaliitto, byrokratiaa ja joskus myös liikunta\-tapahtumia
opiskelijoille.

\textbf{OPARI}: Synonyymi sanalle opintopiste.
Ks.\,Opintopiste.

\textbf{OPETUTOR}: Opettajatutor, omaopettaja.
Opiskelijalle nimetty nimikko-opettaja, jonka on tarkoitus antaa henkilökohtaista
opintoneuvontaa ja olla opiskelijan
ensimmäinen kontakti kaukaiselta ja
arvokkaalta vaikuttavaan henkilökuntaan.

\textbf{OPINTOPISTE}: Nykyään käytettävä
kurssien ja tutkintojen mitta. Yksi opintopiste
vastaa 27~tunnin työtä, 60~opintopistettä
vastaa lukuvuoden työmäärää tai 1600~tuntia opiskelutyötä.

\textbf{OPINTOPUTKI}: Tämän ja akateemisen
vapauden välillä sinun tulee keplotteleman,
kunnes valmistut tuottavaksi työntekijäyksiköksi.
Myös nimitys kävelytunnelille,
joka johtaa Porthanian kulmalta Kaisaniemen
metroasemalle.

\textbf{OPINTOREKISTERIOTE}: Vrt.\,tiliote,
listaus opintoviikkosaldostasi. WebOodista
saa epävirallisia, viralliset paperit saa kerran
lukukaudessa hakea opintotoimistosta
ilmaiseksi.

\textbf{OPISKELIJAHUONE}: Opiskelijoiden
käytössä oleva tila. Voi lukea,
laskea, jutella, pelata, keittää kahvia, nukkua.
Matemaagikoilla Komero, tilasto\-ihmisillä Survomo, käpistelijöillä
Gurula, kemisteillä Opsos, geologeilla
Kasvis ja maantieteilijöillä sohvat ja Suppa. Fyysikaalisilla tieteillä edelleen
nimeämätön (puhekielessä ``OH'') tila~E120 (e-koodi karmiini) Physicumissa.

\textbf{OPISKELIJAKORTTI}: YO-kortti,
SYL-kortti, UniCard, Lyyra ja nykyisin
Frank. Antaa alennuksia, päästää sisään
ym.\,mukavaa.

\textbf{OPM}: Oma Pullo/Pyyhe Mukaan. Myös
Opetusministeriö.

\textbf{OPPONENTTI}: Vastaväittäjä. Seminaareissa
ja tohtorinväittäjäisissä esiintyvä laji.

\textbf{OPSOS}: Kemistien opiskelijahuone.

\textbf{OPTIO}: Johdannaissopimus, jossa option myyjä eli asettaja antaa sitovan lupauksen kaupan tekemisestä jollakin kohde-etuudella sovittuna hetkenä tai ajanjaksona tulevaisuudessa tiettyyn hintaan. Vaihtoehtoisesti maisteriohjelma, johon voi siirtyä suoraan kanditutkinnon suorittamisen jälkeen.

\textbf{OVT}: Oma Vapaa Tahto. Odottaa iltaisin kotona.
Esim.\,lähteä kesken illan kotiin ``omasta
vapaasta tahdostaan'', olla dokaamatta
``omasta vapaasta tahdostaan'' jne.

\textbf{PERUNATORI}: YO-aukio. Uuden ja
Vanhan yo-talon välissä oleva aukio.

\textbf{PHYSICUM}: Kumpulassa sijaitseva
luonnontieteen kehto, jossa majailevat mm.\,fysikaalisten tieteiden, maantieteen ja geologian
osaajat. Katolta hyvät näköalat, jos
sinne joskus pääsee.

\textbf{PIENRYHMÄOHJAUS}: Tuutorointi.
Yliopiston ainejärjestöiltä (alunperin Limekseltä
ja teologeilta) omima toiminta,
jolla helpotetaan uuden opiskelijan sopeutumista
uuteen opiskeluympäristöönsä.

\textbf{PK}: Peruskurssi, fyssan, käpistelyn tai muun.

\textbf{PORTSU}: Porthania. Yliopistonkatu 3.
Pidetään luentoja, tenttejä. Sisältää myös
kuppilan, paperikaupan, opiskelijatyönvälityksen,
humanisteja, juristeja ym.\,ihmeellistä.

\textbf{PROMOOTIO}: Pramea monipäiväinen
juhlatilaisuus, jossa tutkinnon suorittaineita
vihitään juhlallisesti maistereiksi ja tohtoreiksi.

\textbf{PRUJU}: Kasa monisteita jostakin luennosta.
Saa silloin tällöin ostaa valmiina pakettina,
usein täytyy tyytyä itse kopioimaan.

\textbf{PUBLIIKKI}: Valmistumisjuhlan vanha nimitys. Karu tilaisuus, jossa saat kukan
käteesi.

\textbf{PÄÄRAKENNUS}: Pitää sisällään ruokalan,
kahvilan, luentosaleja, opintoneuvonnan
ym.\,toimistoja, humanisteja, rehtorin
ym.\,hallitsijoita.

\textbf{RATKOMO}: Exactumin 3.\,kerroksen
käytävillä ja Komeroa lähellä olevassa luokassa
oleva matemaatikkojen laskualue,
johon voi tulla ratkomaan tehtäviä tuttujen
kanssa ja kysyä apua kinkkisiin pulmiin
ohjaajilta ja muilta kanssaopiskelijoilta.

\textbf{RESONANSSI}: Systeemin valikoiva reagointi
tietyillä taajuuksilla annettuihin impulsseihin.
Fyysikkojen oma ainejärjestö.

\textbf{RIIPPARIT}: Riippukansiot. Joskus Maantieteen laitokselta löytyi henkilökohtaiset kansiot, joiden tarkoitus
oli helpottaa maantieteen opiskelijoiden
keskinäistä yhteydenpitoa.

\textbf{SAUNAILTA}: Illanvietto, joskus ohjelmallinen,
joskus ei. Ei ole pakko saunoa,
mutta saa; yleensä yhteissaunassa (miesten/
naisten vuorotkin löytyy).

\textbf{SILTAVUORENPENGER~20}: Käyt\-täy\-tymis\-tie\-tei\-li\-jöi\-den
koti.

\textbf{SISU}: Hellittämätöntä tahdonvoimaa, lannistumattomuutta. (Sitä tämän työkalun käyttämiseen todella tarvitaan.)

\textbf{SITSIT}: Akateeminen pöytäjuhla, jossa
lauletaan (juoma)lauluja, pidetään hauskaa
ja jossain välissä vielä yritetään nauttia kolmen
ruokalajin illallinen. Pukukoodi vaihtelee
virallisesta haalareihin.

\textbf{SIVISTYS}: Kasvatuksen tietä omaksuttu
tieto ja henkinen kehittyneisyys. Myös kattosauna
Domus Gaudiumilla.

\textbf{SOHVAT}: Maantieteen opiskelijoiden
epävirallinen opiskelijatila Physicumin valopihalla.

\textbf{SOOL}: Suomen Opettajaksi Opiskelevien
Liitto.

\textbf{SPEKTRUM}: Ruotsinkieliset matematiikan,
fysiikan, kemian ja tietojenkäsittelyn
opiskelijat. Kerhohuone Klubben Kirkkokadulla.

\textbf{SPORA}: (Ruots.\,spårvagn) Kiskogiljotiini,
raitiovaunu (engl.\,tram). Janoisille
myös Spårakoff.

\textbf{STEISSI}: Assa. Rautatieasema, makkarataloa
vastapäätä.

\textbf{STEVARI}: Vartija, yleensä Kaivopihan
ympäristössä. Käy aina silloin tällöin Uudella
kyselemässä vastuuhenkilöitä ja
muuta mukavaa. Tavallisesti kuitenkin varsin
kiltti ellet ala ryppyilemään. Huhutaan
avanneen ovia ja vaihtaneen sulakkeita joskus
bileiden aikana.

\textbf{STOKKA}: Stocka, Stockmann. Pohjoismaiden
suurin tavaratalo aivan Helsingin
keskustassa.

\textbf{STUDIA GENERALIA}: Tavalliselle rahvaallekin
tarkoitettu luentosarja; yliopiston
parasta antia.

\textbf{SUMA}: Suunnittelumaantiede.

\textbf{SUPER}: Loistava, mahtava, ihmeellinen.
Myös Teor.\,Fys.\,kurssi Suhteellisuusteorian
perusteet.

\textbf{SYKLOIDI}: Käyrä, joka syntyy pitkin
koordinaattiakselia vierivän ympyrän kehältä
valitun pisteen piirtämänä. Limeksen
virallinen äänenkannattaja, ilmestyy $n$ kertaa
vuodessa. Saa lukea, kirjoittaa ja kuvittaa
vapaasti ja ilman sensuuria (?). Luettavissa
osoitteessa \url{www.limes.fi/sykloidi}.

\textbf{SYL}: Suomen ylioppilaskuntien liitto.
HYYn ja Eduskunnan välinen porras hyrokraateille.

\textbf{SYMBIOOSI}: Kahden eliön molempia
hyödyttävä yhteistoiminta. Myös Helsingin
yliopiston biologien ainejärjestö.

\textbf{SYNOP}: Meteorologian opiskelijoiden ainejärjestö.
Pieni mutta pippurinen.

\textbf{TEDDY}: Söpö ja pehmoinen nallekarhu.
(Kem sl.) Termodynamiikka ja dynamiikka,
jaettu nykyään Termoon ja Dynyyn.

\textbf{TENTTI}: Joukkokokous, jossa yritetään
saada selville, kuka on lukenut tarkimmin
oppikirjojen petiitti\-osuudet ja opetellut
sivunumerot sekä kuvatekstit. Yleensä
väitetään tenttien tuloksilla olevan korrelaatiota
opinto\-menestyksen kanssa, mutta
vaihto\-ehtoisiakin tulkintoja on esiintynyt.
Tentti\-tuloksista merkittävin variaabeli on
totuusarvo (hyväksytty/hylätty), muita tulkintoja
voidaan pitää nykyisen kilpailu\-yhteis\-kunnan
mukanaan\-tuomana epä\-terveenä
vääristymänä.

\textbf{TIEDEKUNTANEUVOSTO}: TDK-neuvosto,
erit.\,ML-tiedekunnan. Täällä
pyörivät astetta isommat rattaat kuin koulutus\-ohjelmien
johtoryhmissä.

\textbf{TIETOTEKNIIKKAOSASTO}: Helsingin
yliopiston ATK-osasto tarjoaa tietotekniikkaa
ensisijaisesti opetus-, tutkimus- ja
opiskelutarkoituksiin. Isoja ja pieniä tietokoneita
kaikkien opiskelijoiden käytettävissä.
Katso myös ATK-asema, Kolikot.

\textbf{TIRA}: (TKT sl.) Tietorakenteet. TKT:n
kurssi.

\textbf{TKO-äly}: Se ainoa oikea käpistelijöiden
oma järjestö. Liity ja innostu. Excursioita,
saunailtoja, tempauksia ja vaikka mitä. Ks.\,Gurula.

\textbf{TKT}: Tietojenkäsittelytiede. Aine, jota datagurut
todella opiskelevat.

\textbf{TODARI}: (Mat.sl.) TN. Todennäköisyyslaskenta.

\textbf{TOIMISTO}: Limeksen toimisto, Exactum~C132. Toimistolta voit ostaa mm.\,oppikirjoja
ja haalarimerkkejä tai vaikka kahvitella.
Aukioloajat epäsäännölliset, mutta auki
lähes päivittäin.

\begin{figure*}[!b]
	\centering
	\includegraphics[width=0.5\textwidth]{lammas.png}
\end{figure*}
\textbf{TOPO}: (Mat.sl.) Topologia.

\textbf{TRIVIAALI}: Itsestään selvä. Se kohta
todistuksesta, todistuksessa tms., jonka
ymmärtävät muutkin kuin luennoitsija tai
päinvastoin. Sanaa ei pidä käyttää tenttivastauksessa.

\textbf{TORVALDS, LINUS}: Entinen TKT:n laitoksen
assistentti, nykyisin USA:ssa. Linus
on Linux-käyttöjärjestelmän isä. Ilmielävä
suuren luokan Guru. Limeksen ja TKO-älyn
kunniajäsen.

\textbf{TUNNELOITUMINEN}: (Fys.sl.) Hyödyllinen
taito esim.\,ruuhkabussissa.

\textbf{TUOMIOPÄIVÄ}: Muille se on 24.4.,
fyysikoille kaksi kuukautta myöhässä. Etsi
vinkkejä fysikaalisten tieteiden yhteisestä
laulukirjasta.

\textbf{TUUTORI}: Pienryhmäohjaaja. Vanhempi
opiskelija, joka auttaa uudet fuksit alkuun
yliopistouralla. Muista ostaa tuutorillesi
juoma!

\textbf{TUUTOROITAVA}: Tuutorlapsi = uusi
innokas opiskelija, joka ottaa osaa pienryhmäohjaukseen.
Ks.\,fuksi.

\textbf{TUUTORRYHMÄ}: Vanhemmasta (tuutori)
ja $n$~kpl uudesta opiskelijasta (tuutoroitava)
koostuva pienryhmä, jonka tarkoituksena
on tutustuttaa uudet opiskelijat
yliopistoon ja opiskelijaelämään ja pitää
muutenkin hauskaa.

\textbf{ULKKARI}: Ulkomaalaisopiskelija.

\textbf{ULLANLINNANMÄKI}: Kaivarin eteläosan
mäki, jossa on URSAn tähtitorni ja
1.5.\,paljon samppanjapullonkorkin näköisiä
ufoja.

\textbf{UNICARD}: mm.\,opiskelija- ja kirjastokorttina
toimiva älykortti. Tavataan vanhemmilla
opiskelijoilla, uudet opiskelijat
saavat Frankin.

\textbf{UNIGRAFIA}: Yliopiston sisäiseen painotoimintaan
keskittynyt painolaitos Vuorikadulla,
Meilahdessa ja Viikissä.

\textbf{URAPALVELUT}: Rekry. Palveluita
työnhakuun, harjoitteluun, uraan ja muuhun
työelämään liittyvään.

\textbf{URHO}: Klusterin oma maskotti. Saa halata,
mutta ei roikkua.

\textbf{UUSI}: Uusi ylioppilastalo (``Uusi'' = rak.\,1910!), joskus kuullaan käytettävän myös
nimitystä Osakuntatalo.

\textbf{UUSI YLIOPPILASTALO}: Mansku
5 A--C. Löytyy mm.\,Alina, Mannerheim,
HYY\dots

\textbf{VAIHTOEHTOINEN}: Vaihtoehtona oleva, eli siis pakollinen.

\textbf{VALINNAINEN}: (1) Sellainen, jonka voi jättää valitsematta. (2) Pakollinen.

\textbf{VALTSIKA}: Valtiotieteellinen tdk. Unioninkatu~37. Klassinen miljöö ja UniCafé.

\textbf{VANHA}: Vanha ylioppilastalo, Mansku~3.
Uuden vieressä: juhlasali, musiikkisali, yritystapahtumia,
bileitä, klubeja, vuosijuhlia,
Kuppila ja terassi. Aikanaan myös opiskelijoiden
hengauspaikka.

\textbf{VASARA}: Tylppä, varrellinen lyömätyökalu,
astalo. Myös Helsingin yliopiston
geologian opiskelijoiden ainejärjestö.

\textbf{VIIKKI}: Biotieteiden, farmasian ja maatalousmetsätieteellisen
kampus pellon reunalla.

\textbf{YHTEISLÄHTÖ}: Kokoontuminen sovittuun
paikkaan excursiolle, matkalle tai johonkin
tapahtumaan lähtöä varten. Yleensä
1/2--2~tuntia ennen lähtöä, ennen vaativia urheilusuorituksia ym.\,jopa 1~vrk etukäteen.

\textbf{YLEISKOKOUS}: Vähintään 2~kertaa
vuodessa kokoontuva erit.\,Limeksen ylin
päättävä elin: päättää toiminnan suuntaviivoista
ja hallituksesta, sisältää asiallista
keskustelua, päättyy joskus illanviettoon.
Hoitaa sääntömääräiset asiat.

\textbf{YLIOPISTOKIRJAKAUPPA}: Suomalaisen
kirjakaupan omistama tieteellisten
kirjojen kauppa mm.\,Keskustassa ja Viikissä.
Saa kaikenmaailman kirjoja joko suoraan
hyllystä tai tilaamalla.

\textbf{YLIOPPILASAUKIO}: Ks.\,Perunatori.

\textbf{YLIOPPILASLEHTI}: Tipahtaa säännöllisesti
jokaisen opiskelijan postiluukusta.
Luettava ja hyödyllinen. Toisinaan paskajuttuja.

\textbf{YLKKÄRI}: Ks.\,Ylioppilaslehti.

\textbf{YO}: Rap-artistien tervehdyssana. Yliopisto,
ylioppilas, ym.\,yli-o\dots

\textbf{YTHS}: Ylioppilaiden terveydenhoitosäätiö.
Töölönkatu~37~A: hyvää ja halpaa
terveydenhoitoa, pakollinen terveydenhoitomaksu
ei ole turhaa kiskontaa. Myös kaikilla
muilla yliopistopaikkakunnilla.

\textbf{ÄMMÄLÄ}: (arkaainen) Matemaattis-luonnon\-tieteellinen
tiedekunta, ML. Erit.\,tiedekunnan opinto\-toimistoa kutsutaan
Ämmäläksi.
\end{document}